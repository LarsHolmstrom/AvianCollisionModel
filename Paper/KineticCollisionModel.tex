\documentclass[10pt,conference]{IEEEtran}
\usepackage{amsmath}
\usepackage{amssymb}
\usepackage{bm}
\usepackage{graphicx}

%>>>>>>>>>>>>>>>>>>>>>>>>>>>>>>>>>>>>>>>>>>>>>>>>>>>>>>>>>>>>>>>>>>>>>>>>>>>>>>
\begin{document}

\title{Modeling Avian-Turbine Collisions: An Angle Independent Extension to the Tucker Model}
\author{Lars Holmstrom, Hamer Environmental L.P.}
\maketitle

%==============================================================================
%Abstract
%==============================================================================
\begin{abstract}
A mathematical model is described which can estimate the probability of a collision between a bird passing through a
wind turbine and one of the rotors of the turbine. The model improves upon the often used ``Tucker Model" in that it
accounts for different angles of approach other than perpendicular or parallel to the rotor plane. The model makes a
number of assumptions for the sake of analytic simplicity, most notably the exclusion of any behavioral interaction
between the bird and the turbine. It is useful, however, as a building block for more comprehensive models which may
account for these behavioral effects and the interactions of multiple turbines in wind farms.
\end{abstract}

%==============================================================================
\section{Introduction}
%==============================================================================
Wind power is quickly becoming an attractive renewable energy source across the globe due its minimal emissions
production amid the rising awareness of global warming. Wind power is not, however, without environmental impact. It
has been shown that wind turbines pose a threat to bird populations resulting from direct collisions with the rotor
blades or turbine towers \cite{Johnson2002}\cite{Johnson2004}. This effect has been documented for numerous
installations, including the Altamont Pass Wind Resource Area (APWRA), where it is estimated that somewhat more than
1,000 birds are being killed annually by about 5,000 wind turbines \cite{Thalender2003}. While this is an atypically
high mortality rate resulting from the local geology, weather, and behavior of the birds, it has nonetheless made it
clear that wind farms may have a significant impact on bird populations.

Due to the potential danger to endangered and/or protected bird species and the knowledge that the location of the
turbines with respect to these populations plays an important role in the resulting mortality rate, it has become
common to assess the impact of proposed wind farms on local and migratory bird populations before they are even built.
While other factors such as species and resource displacement may also result from wind farm construction, a key
component of the impact assessment is an estimation of the annual bird mortality rate resulting directly from rotor and
tower collisions.

This paper discusses a model based approach to this estimation problem. The inputs into the model are based on the
characteristics of the wind turbine along with information about wind conditions and the flight paths taken by the
birds. The output is an estimated probability of collision for a specified flight path based upon the kinetics of the
moving bird and the rotating wind turbine. This paper will focus on the interaction between a single bird and turbine.
This result can be extrapolated to estimate collision probabilities for populations of birds flying across wind farms.

The model described in this paper is not revolutionary in that it proposes a new way to approach the mortality
estimation problem. In fact, it is based on the collision model proposed by V. A. Tucker in his 1996 paper ``A
Mathematical Model of Bird Collisions With Wind Turbine Rotors" \cite{Tucker1996}. This model has become the basis for
a number of other models which address the larger problem of mortality estimation for an entire wind farm
\cite{Cooper2004}\cite{Madders2006}. Where \cite{Tucker1996} provides solutions for the cases where the bird's air
velocity is either parallel or perpendicular to the plane of the turbine rotors, the model described in this paper
solves the more general case of oblique approach angles. This is an important improvement, since birds often follow
flight paths that are not based on the wind direction, but rather on geologic features such as `inland' or along a
coastline.

%==============================================================================
\section{Methods}
%==============================================================================
\subsection{Model Assumptions/Simplifications}
In order to simplify the analysis of the problem, a number of assumptions have been made in the design of the model.
Most of these have been inherited directly from the Tucker model, but the most significant of these warrant discussion.
While some of these simplifications limit the generality and accuracy of the model, they can be justified in that they
make the derivation and implementation of the model more tractable.

One of the assumptions shared by both the Tucker and the present model is that birds have a straight flight path which
is parallel to the ground. Since birds engaged in other behaviors such as foraging and hunting often use erratic or
circular flight patterns which vary in altitude \cite{Thalender2003}, the model will have to be further adapted to
apply to these situations. This extension is beyond the scope of this paper.

Another simplification shared by both models is the decision to model a flying bird as a two dimensional rectangle
representing the wingspan and length of the bird. Clearly, a bird has girth as well, may be flapping its wings while
traversing the plane of the turbine, and is often more of a ``T" shape than a rectangle when viewed from above. This
decision was made to simplify the mathematical analysis of the problem and to reduce the description of the bird's
physical shape down to two parameters. These simplifications, as opposed to a more complete description of the physical
dimensions of a bird in flight, can be shown to have only a small effect on the outcome of the model.

The third and most profound assumption in the current model is that there are no behavioral interactions between the
birds and the turbine. Since bird species have been shown to avoid turbines and wind farms
\cite{Kahlert2003}\cite{Chamberlain2006}, this simplification will clearly bias the model towards over-estimating the
collision probabilities. The Tucker model makes an attempt to address this interaction by defining a rotor speed,
$v_\circ$, below which a bird can avoid the moving rotor. While the motivation behind this addition to the model is
sound, it is not clear how $v_\circ$ should be estimated for different bird species. Furthermore, it is clearly not a
complete or accurate description of the behavioral interaction in question, since there is evidence that birds hit the
non-moving components of the wind turbine, including the tower. Since these behavioral interactions are incompletely
addressed in the Tucker model, the model proposed in this paper does not include them. What the model instead provides
is an upper bound on the collision probabilities when a straight flight path and no behavioral interaction is assumed.
Behavioral corrections to the outputs of this model are also beyond the scope of this paper.

Finally, the last significant assumption that this model makes concerns the collision probabilities that are calculated
when a three-dimensional rotor blade is considered (one that has a chord length and twist angle along its length).
Similar to the Tucker model, the chord length and pitch is assumed to be constant in the region where the bird passes
through the rotor plane. In this model, the chord and pitch that are used are calculated from the rotor position
where the nose of the bird passes through rotor plane, even though the bird may drift to the right or left depending
on the angle of flight relative to this point of entry. This simplification can be shown to have an insignificant
affect on the outcome and greatly simplifies the collision analysis when a three-dimensional rotor is considered.

\subsection{Model Parameters}
\begin{table}
  \centering
%  \textbf{User Specified Model Parameters}\\
  \vspace{1 mm}
  \begin{tabular}{|ll|}
  \hline
  \textbf{Bird Parameters} &\\
  $w$ & Wing Span (m)\\
  $l$ & Length of Bird (m)\\
  $V_b$ & Bird's Air Speed (m/s)\\
  $\phi$ & Angle of Approach (degrees)\\
  \hline
  \textbf{Turbine Parameters} & \\
  $N$ & Number of Rotors\\
  $R$ & Radius of Turbine (m)\\
  $r$ & Radius of Hub (m)\\
  $\Omega$ & Angular Velocity (rpms)\\
%  $f(c)$ & Chord Length Function (m)\\
%  $f(\beta)$ & Chord Angle Function (degrees)\\
  \hline
  \textbf{Wind Parameters} &\\
  $V_w$ & Velocity\\
  $a$ & Axial Induction Factor\\
  \hline
  \end{tabular}
  \caption{User Specified Model Parameters}
  \label{table.user_parameters}
\end{table}


Each application of the model will correspond with a turbine design and a bird species under consideration. In
addition, the results of field surveys will provide critical site dependent information about the wind conditions and
the bird's local flight paths. Table \ref{table.user_parameters} lists the user specified model parameters and their
definitions. The angle of approach describes the bird's rotational orientation with respect to downwind ($0^\circ$).
Rotation to the left results in negative angles and rotation to the right results in positive angles. This is not
generally equal to the angle that the bird's flight path intersects the rotor plane, since flight direction is
dependent on both the bird and wind velocity. See Fig. \ref{fig.Plus85} for an example. In this figure, the red lines
indicating the flight path of the bird are clearly not parallel to the angle of approach of the bird. The axial
induction, $a$, accounts for the fact that air flowing through the disk has a speed less than the wind speed
\cite{Wilson1994}.
The chord length function and the chord angle function describe the rotor chord length and twist angle as a function of
the radius of the rotor.

For the purposes of comparison, the chord functions described in the Tucker Model are used in the results published in
this paper. Table \ref{ChordFunctions} defines these functions and Fig. \ref{fig.ChordCharacteristics} shows plots of
these functions.
\begin{table}
  \centering
  \textbf{Chord Length (c) and Angle ($\beta$) Functions}\\
    \begin{tabular}{|lllll|}
  \hline
    % after \\: \hline or \cline{col1-col2} \cline{col3-col4} ...
    $c$ & $=$ & $0$ & for & $r/R< 0.1$ \\
    $c$ & $=$ & $0.115R$ && $0.1 < r/R < 0.25$ \\
    $c$ & $=$ & $0.168R-0.240r+0.1r^2/R$ && $0.25 \leq r/R < 1$ \\
    &&&&\\
    \multicolumn{5}{|l|}{$\beta = 0.640 \arctan(8(r/R-0.015))^{-1}-0.073$}\\
  \hline
    \end{tabular}
  \caption{This is a table}\label{ChordFunctions}
\end{table}

\begin{figure}
   \centering
   \includegraphics[width=1.0\columnwidth]{ChordCharacteristics}
   \caption{}
   \label{fig.ChordCharacteristics}
\end{figure}

\subsection{General Algorithm For Calculating Collision Probabilities}
When considering a linear bird flight path through the rotor plane, a straightforward algorithm can be used to
calculate the probability of a collision with one of the rotor blades:
\begin{enumerate}
  \item As a bird passesthrough the rotor plane for a given flight path, calculate the full range of angles the turbine rotors can take on that will result in a collision with the bird.
  \item Divide this composite range by 360 degrees to generate a probability of collision with one of the rotors.
\end{enumerate}
While this general algorithm is simple, the calculation of the rotor ranges can be complicated.

\begin{figure}[h!]
   \centering
   \includegraphics[width=1.0\columnwidth]{DeadOn}
   \caption{\textbf{Top Pane:} A bird at the instant it enters the rotor plane for an approach angle of $0^\circ$ (dead-on). The bird's air velocity $V_b$ is 5
m/s, the wind velocity $V_w$ is 10 m/s, and the other model parameters are as specified in Table
\ref{table.example_parameters}. The green circle indicates the position of the rotor that will collide with corner 2 of
the bird at the instant the bird passes through the rotor plane. The black circle indicates the position of the rotor
that will collide with corner 4 as the bird exits the rotor plane. \textbf{Middle Pane:} A view of the area swept out
by the turbine rotors looking downwind as the bird enters the rotor plane. \textbf{Bottom Pane:} A view of the top
right quadrant of the area swept out by the turbine rotors indicating the variables used in the discussion of the
model.}
   \label{fig.DeadOn}
   \end{figure}

\begin{figure}
   \centering
   \includegraphics[width=1.0\columnwidth]{DeadOnTurbine}
   \caption{Probability contours for a downwind flight path where the bird's air velocity $V_b$ is 5
m/s, the wind velocity $V_w$ is 10 m/s, and the other model parameters are as specified in Table
\ref{table.example_parameters}}
   \label{fig.DeadOnTurbine}
   \end{figure}

\begin{figure}
   \centering
   \includegraphics[width=1.0\columnwidth]{DeadOnTurbineUpwind}
   \caption{Probability contours for an upwind flight path where the bird's air velocity $V_b$ is -10
m/s, the wind velocity $V_w$ is 5 m/s, and the other model parameters are as specified in Table
\ref{table.example_parameters}}
   \label{fig.DeadOnTurbineUpwind}
   \end{figure}

\subsection{Analysis of Downwind and Upwind Flight Path with 1D Rotor}
This is the most basic scenario to analyze and provides a good opportunity for an introduction into the collision
probability calculations described in this paper. The middle pane of Fig. \ref{fig.DeadOn} depicts a wind turbine as it
appears looking downwind. The plane of the rotor blades is perpendicular to the wind direction, as is the case in
normal operation. The origin of a left-handed 3 dimensional (xyz) coordinate space is located at the center of the
turbine hub. The z-component increases in the up direction, the y-component increases to the right, and the x-component
increases downwind and is negative at the figure viewpoint. The turbine blades rotate counter clockwise when looking
downwind. The horizontal blue bar depicts the position of the bird as it enters the rotor plane. Since the flight path
is parallel to the wind direction, the $x$ and $y$ components of the bird's groundspeed are given by
\begin{equation*}
    V_{bx} = V_b +(1-a)U
\end{equation*}
and
\begin{equation*}
    V_{by} = 0
\end{equation*}


\subsubsection{Calculating Collision Points}
The top pane of the figure is a top-down view of the bird at the instant it enters the rotor plane. The green circle
indicates the most advanced position of the rotor on the y-axis that can produce a collision with the bird, $y_+$. The
earliest collision at this point takes place as soon as the bird enters the rotor plane as corner 2 of the bird clips
the rotor as it is rotating away to the left. The black circle indicates the least advanced position of the rotor on
the y-axis that can produce a collision with the bird, $y_-$. The latest collision at this point takes place as soon as
the bird exits the rotor plane as the rotor clips corner 4 of the bird. In the case of downwind flight (flight path
perpendicular to the rotor plane), the y-components of the rotor positions corresponding to these two points are given
by:
\begin{equation*}\label{DeadOnLeft}
    y_{+} = y_{\text{nose}} - w/2
\end{equation*}
and
\begin{equation*}\label{DeadOnRight}
    y_{-} = y_{\text{nose}} + w/2
\end{equation*}
where $y_{\text{nose}}$ is the y coordinate of where the nose of the bird crosses the rotor plane (at 5m in Fig.
\ref{fig.DeadOn}), and $w$ is the width of the bird (1m in Fig. \ref{fig.DeadOn}).

\subsubsection{Rotor Angle Resulting in Collision}
The lower pane of Fig. \ref{fig.DeadOn} shows view of the top right quadrant of the area swept out by the turbine
rotors indicating the variables used in the following discussion of the model. Rotor rotational angles, $\theta_+$ and
$\theta_-$, can be calculated which correspond to the collision points $y_+$ and $y_-$. These angles are given by:
\begin{equation*}
    \theta_+ = \arctan(z/y_+)
\end{equation*}
and
\begin{equation*}
    \theta_- = \arctan(z/y_-)
\end{equation*}
The time it takes corner 4 of the bird to intersect the rotor plane is given by $l/V_{bx}$, where $l$ is the length of
the bird. The rotor can rotate $(l/V_{bx})\Omega$ degrees in this period of time. We can therefore calculate the
initial angle of the rotor which will clip corner 4 upon the bird's exit from the rotor plane:
\begin{equation*}
    \theta_x = \theta_- - (l/V_{bx})\Omega
\end{equation*}
The complete range of rotor angles, $\theta_r$, that would result in a collision as the bird passes through the turbine
plane is given by:
\begin{eqnarray}
\nonumber    \theta_r &=& \theta_+ - \theta_x\\
             &=& \arctan(z/y_+) - \arctan(z/y_-) + (l/V_{bx})\Omega
\end{eqnarray}
These results agree with those described in \cite{Tucker1996} where $\arctan(z/y_+) - \arctan(z/y_-)$ is equal to
$(\Delta\Psi)_s$ and $(l/V_{bx})\Omega$ is equal to $(\Delta\Psi)_x$. A slightly different approach has been taken in
this analysis to aid the calculation of the collision probabilities for oblique angles of approach.

\begin{table}
  \centering
  \label{table.example_parameters}
%  \textbf{Model Parameter Values Used In Examples}\\
  \vspace{1 mm}
  \begin{tabular}{|ll|}
  \hline
  \textbf{Bird Parameters} &\\
  $w$ & 1 m\\
  $l$ & 0.5 m\\
  $V_b$ & variable\\
  $\phi$ & variable\\
  \hline
  \textbf{Turbine Parameters} & \\
  $N$ & 3 rotors\\
  $R$ & 10 m\\
  $r$ & 0.5 m\\
  $\Omega$ & 72 rpms\\
%  $f(c)$ & Chord Length Function (m)\\
%  $f(\beta)$ & Chord Angle Function (degrees)\\
  \hline
  \textbf{Wind Parameters} &\\
  $V_w$ & variable\\
  $a$ & 0.25\\
  \hline
  \end{tabular}
  \caption{Model Parameter Values Used In Examples}
\end{table}

\subsubsection{Collision Probability Calculation}
Provided the bird, turbine, and wind characteristics (see Table \ref{table.user_parameters}), and a point $(0,y,z)$
where the nose of the bird intersects the rotor plane, we can calculate the probability of a collision given a downwind
flight path and a 1 dimensional rotor:
\begin{eqnarray}\label{DownwindProbability}
    \nonumber P(y,z) &=& \frac{360^\circ}{N\theta_r}\\
    &=& \frac{360^\circ}{N(\theta_+ - \theta_- + (l/V_{bx})\Omega)}\\
    \nonumber &=& \frac{360^\circ}{N(\arctan(z/y_+) - \arctan(z/y_-) + (l/V_{bx})\Omega)}
\end{eqnarray}

Fig. \ref{fig.DeadOnTurbine} shows the resulting turbine probability contours where the bird's air velocity $V_b$ is 5
m/s, the wind velocity $V_w$ is 10 m/s, and the other model parameters are as specified in Table
\ref{table.example_parameters}.

\subsubsection{Upwind Flight Path Analysis}
The analysis of upwind flight is identical to the downwind flight path analysis except that the sign of the bird
velocity parameter $V_b$ is negative instead of positive. When added to the wind velocity (which is always
perpendicular to the rotor plane) and after accounting for $a$, the axial induction factor, this generally results in a
reduced ground velocity for the bird which increases the chance of collision with a rotor.

Fig. \ref{fig.DeadOnTurbineUpwind} shows the resulting probability contours where the bird's air velocity $V_b$ is -10
m/s, the wind velocity $V_w$ is 5 m/s, and the other model parameters are as specified in Table
\ref{table.example_parameters}.

\section{Analysis of Oblique Flight Path with 1D Rotor}
This analysis includes all angles of approach that the bird may take towards the wind turbine -- not just directly
downwind or upwind as the Tucker model addresses. Oblique angles of approach require additional care in the calculation
of the first and last possible collision points between the bird and the rotor. This paper will cover the 4 different
collision scenarios based on different angles of approach. For each of these examples, the bird's air velocity $V_b$ is
5 m/s, the wind velocity $V_w$ is 10 m/s, the point where the bird's nose intersects the rotor plane is $(x,y,z) =
(0,5\text{m},5\text{m})$, and the other model parameters are as specified in Table \ref{table.example_parameters}.
Since the bird is no longer traveling parallel to the wind (and perpendicular to the rotor plane), there are now $x$
and $y$ components of the bird's velocity, $V_{bx}$ and $V_{by}$. These are given by
\begin{equation*}
    V_{bx} = V_b cos(\phi)+(1-a)U
\end{equation*}
and
\begin{equation*}
    V_{by} = V_b sin(\phi)
\end{equation*}


\begin{figure}
   \centering
   \includegraphics[width=1.0\columnwidth]{Minus7}
   \caption{\textbf{Leading Edge Collision.} A bird at the instant it enters the rotor plane for an approach angle of $-7^\circ$. The bird's air velocity $V_b$ is 5
m/s, the wind velocity $V_w$ is 10 m/s, and the other model parameters are as specified in Table
\ref{table.example_parameters}. The green circle indicates the position of the rotor at this instant that will clip
corner 2 of the bird as the bird passes through the rotor plane. Any rotor position between this point at the corner 1
of the bird will result in a leading edge collision. The black circle indicates the position of the rotor that will
clip corner 4 of the bird as it passes through the rotor plane. This is the last point the bird can collide with the
rotor since no trailing edge collision is possible in this case.}
   \label{fig.Minus7}
   \end{figure}


\begin{figure}
   \centering
   \includegraphics[width=1.0\columnwidth]{Minus15}
   \caption{\textbf{Trailing Edge Collision.} A bird at the instant it enters the rotor plane for an approach angle of $-15^\circ$. The bird's air velocity $V_b$ is 5
m/s, the wind velocity $V_w$ is 10 m/s, and the other model parameters are as specified in Table
\ref{table.example_parameters}. The green circle indicates the position of the rotor that will collide with corner 1 of
the bird at the instant the bird passes through the rotor plane. No leading edge collision is possible in this case.
The black circle indicates the position of the rotor that will clip corner 3 of the bird as it passes through the rotor
plane. This is the last point the bird can collide the rotor since a trailing edge collision is possible in this case.}
   \label{fig.Minus15}
   \end{figure}

\subsection{Approach Angle in Direction of Turbine Rotation}
In this case, the bird approaches the rotor plane with an approach angle in the direction of the rotor rotation. For
downwind flight paths, this is true for negative angles of approach in the top half of the area swept by the rotors
($z>0$) and positive angles of approach in the bottom half ($z<0$). For upwind flight paths, this is true for for
positive angles of approach in the top half and negative angles in the bottom. Fig. \ref{fig.Minus7} and Fig.
\ref{fig.Minus15} show visualizations of the $z>0$, $\phi<0$ case where the angles of approach are $-7^\circ$ and
$-15^\circ$ and the other parameters are as stated above for the oblique approach angle examples. In these figures,
corner 1 is the first point of contact of the bird with the rotor plane. Looking down from above, the turbine is
rotating to the left. The time at which corner 1 intercepts the plane, $t_1$, is considered to be zero. The time at
which the nose and the other corners intercept the rotor plane are given by
\begin{eqnarray*}
    t_{nose} &=& \frac{-w \cdot sin(\phi)}{2 V_{bx}}\\
    t_1 &=& 0\\
    t_2 &=& \frac{-w \cdot sin(\phi)}{V_{bx}}\\
    t_3 &=& t_2 + \frac{l \cdot cos(\phi)}{V_{bx}}\\
    t_4 &=& \frac{l \cdot cos(\phi)}{V_{bx}}
\end{eqnarray*}
and the y coordinates of these intercepts are given by
\begin{eqnarray*}
    y_1 &=& y_{nose}+\frac{w \cdot cos(\phi)}{2}-t_{nose}V_{by}\\
    y_2 &=& y_{nose}+\frac{w \cdot cos(\phi)}{2}+(t_2-t_{nose})V_{by}\\
    y_3 &=& y_2-l \cdot sin(\phi) + (t_3-t_2)V_{by}\\
    y_4 &=& y_1-l \cdot sin(\phi) + t_4V_{by}
\end{eqnarray*}
The time required for the rotor to travel from $y_1$ to $y_2$ is given by
\begin{equation*}
    t_r = \frac{1}{\Omega}(\arctan(y_2/z)-\arctan(y_1/z))
\end{equation*}
Two cases must be considered to determine the types of collisions possible: $t_r>t_2$ and $t_r<t_2$. These two
scenarios may lead to leading edge ($\overline{12}$) and trailing edge ($\overline{34}$) collisions, respectively.
Right edge ($\overline{14}$) collisions are possible and left edge ($\overline{23}$) collisions are impossible in both
cases.

\subsubsection{Leading Edge Collisions}
For the set of parameters used in Fig \ref{fig.Minus7}, even if the rotor is to the left of corner 1 as the bird enters
the rotor plane, it is possible for the leading edge of the bird ($\overline{12}$) to overtake the rotor, resulting in
a collision. Since $t_r>t_2$, the rotor is traveling to the left at a slower velocity than the leading edge of the bird
is sweeping through the rotor plane. The green circle indicates the initial position of the rotor (at $t=0$) that will
clip corner 2 of the bird as the bird passes through the rotor plane. This point is given by
\begin{equation}
    y_+ = \tan(\arctan(y_2/z)+t_2\Omega)z
\end{equation}
The rotational angle associated with the green circle is $\theta_+$ (see bottom pane of Fig. \ref{fig.DeadOn}) and is
given by
\begin{equation}
    \theta_+ = \arctan\left(\frac{z}{\tan(\arctan(y_2/z)+t_2\Omega)z}\right)
\end{equation}

Since the trailing edge of the bird ($\overline{34}$) also sweeps through the rotor plane faster than the rotor to the
left, it is impossible for the rotor to clip corner 3 of the bird. Therefore, the last possible collision with the bird
occurs at the position of the black circle as corner 4 passes through the rotor plane. This point is given by
\begin{eqnarray}
    \nonumber y_- &=& y_4\\
    &=& y_1-l \cdot sin(\phi) + t_4V_{by}
\end{eqnarray}
The rotation angle associated with $y_-$ is $\theta_-$ (see bottom pane of Fig. \ref{fig.DeadOn}) and is given by
\begin{equation}
    \theta_- = \arctan\left(\frac{z}{y_1-l \cdot sin(\phi) + t_4V_{by}}\right)
\end{equation}
The collision probability can then be calculated using (\ref{DownwindProbability}).

The same analysis applies for $z<0$, $\phi>0$ in downwind flight and $z>0$, $\phi>0$ in upwind flight except the corner
definitions must be swapped to account for the positive angle of approach where corner 2 is the first to intercept the
rotor plane.

\subsubsection{Trailing Edge Collisions} For the set of parameters used in Fig \ref{fig.Minus15}, the increased (in magnitude) angle of
approach decreases the speed at which the leading edge of the bird ($\overline{12}$) sweeps through the rotor plane,
such that the rotor takes less time to travel the horizontal distance swept by this leading edge ($t_r<t_2$). This
makes it impossible for a collision to occur between the rotor and the leading edge of the bird. The green circle
indicates the most advanced position of the rotor as the bird enters the rotor plane that will result in a collision.
This collision will occur with corner 1 of the bird at the instant the bird enters the rotor plane. The point is given
by
\begin{eqnarray}
    \nonumber y_+ &=& y_1\\
    &=& y_{nose}+\frac{w \cdot cos(\phi)}{2}-t_{nose}V_{by}
\end{eqnarray}
The rotational angle associated with the green circle is $\theta_+$ (see Fig. \ref{fig.DeadOn}) and is given by
\begin{equation}
    \theta_+ = \arctan\left(\frac{z}{y_{nose}+\frac{w \cdot cos(\phi)}{2}-t_{nose}V_{by}}\right)
\end{equation}

Since the trailing edge of the bird ($\overline{34}$) also sweeps to the left through the rotor plane slower than the
rotor, it is possible for the rotor to clip corner 3 of the bird. Therefore, the last possible collision with the bird
occurs at the position of the black circle as corner 4 passes through the rotor plane. This point is given by
\begin{eqnarray}
    \nonumber y_- &=& y_3\\
    &=& y_2-l \cdot sin(\phi) + (t_3-t_2)V_{by}
\end{eqnarray}
The rotation angle associated with $y_-$ is $\theta_-$ (see bottom pane of Fig. \ref{fig.DeadOn}) and is given by
\begin{equation}
    \theta_- = \arctan\left(\frac{z}{y_2-l \cdot sin(\phi) + (t_3-t_2)V_{by}}\right)
\end{equation}
The collision probability can then be calculated using (\ref{DownwindProbability}).

The same analysis applies for $z<0$, $\phi>0$ in downwind flight and $z>0$, $\phi>0$ in upwind flight except the corner
definitions must be swapped to account for the positive angle of approach where corner 2 is the first to intercept the
rotor plane.


\begin{figure}
   \centering
   \includegraphics[width=1.0\columnwidth]{Plus15}
   \caption{\textbf{Right Edge Collision.} A bird at the instant it enters the rotor plane for an approach angle of $15^\circ$. The bird's air velocity $V_b$ is 5
m/s, the wind velocity $V_w$ is 10 m/s, and the other model parameters are as specified in Table
\ref{table.example_parameters}. The green circle indicates the position of the rotor that will collide with corner 2 of
the bird at the instant the bird passes through the rotor plane. No left edge collision is possible in this case. The
black circle indicates the position of the rotor that will clip corner 4 of the bird as it passes through the rotor
plane. This is the last point the bird can collide the rotor since a right edge collision is possible in this case.}
   \label{fig.Plus15}
   \end{figure}


\begin{figure}
   \centering
   \includegraphics[width=1.0\columnwidth]{Plus85}
   \caption{\textbf{Left Edge Collision.} A bird at the instant it enters the rotor plane for an approach angle of $85^\circ$. The bird's air velocity $V_b$ is 5
m/s, the wind velocity $V_w$ is 10 m/s, and the other model parameters are as specified in Table
\ref{table.example_parameters}. The green circle indicates the position of the rotor at this instant that will clip
corner 3 of the bird as the bird passes through the rotor plane. Any rotor position between this point at the corner 2
of the bird will result in a left edge collision. The black circle indicates the position of the rotor that will clip
corner 1 of the bird as it passes through the rotor plane. This is the last point the bird can collide with the rotor
since no right edge collision is possible in this case.}
   \label{fig.Plus85}
   \end{figure}

\subsection{Approach Angle in Opposite Direction of Turbine Rotation}
In this case, the bird approaches the rotor plane with an approach angle in the opposite direction of the rotor
rotation. For downwind flight paths, this is true for positive angles of approach in the top half of the area swept by
the rotors ($z>0$) and negative angles of approach in the bottom half ($z<0$). For upwind flight paths, this is true
for for negative angles of approach in the top half and positive angles in the bottom. Fig. \ref{fig.Plus15} and Fig.
\ref{fig.Plus85} show visualizations of the $z>0$, $\phi>0$ case where the angles are $15^\circ$ and $85^\circ$ and the
other parameters are as stated above for the oblique approach angle examples. In these figures, corner 2 is the first
point of contact of the bird with the rotor plane. Looking down from above, the turbine is rotating to the right. The
time at which corner 2 intercepts the plane, $t_2$, is considered to be zero. The time at which the nose and the other
corners intercept the rotor plane are given by
\begin{eqnarray*}
    t_{nose} &=& \frac{w \cdot sin(\phi)}{2 V_{bx}}\\
    t_1 &=& \frac{w \cdot sin(\phi)}{V_{bx}}\\
    t_2 &=& 0\\
    t_3 &=& \frac{l \cdot cos(\phi)}{V_{bx}}\\
    t_4 &=& t_1 + \frac{l \cdot cos(\phi)}{V_{bx}}
\end{eqnarray*}
and the y coordinates of these intercepts are given by
\begin{eqnarray*}
    y_1 &=& y_{nose}+\frac{w \cdot cos(\phi)}{2}+(t_1-t_{nose})V_{by}\\
    y_2 &=& y_{nose}-\frac{w \cdot cos(\phi)}{2}-t_{nose}V_{by}\\
    y_3 &=& y_2-l \cdot sin(\phi) + t_3V_{by}\\
    y_4 &=& y_1-l \cdot sin(\phi) + (t_4-t_1)V_{by}
\end{eqnarray*}
The time required for the rotor to travel from $y_2$ to $y_3$ is given by
\begin{equation*}
    t_r = \frac{1}{\Omega}(\arctan(y_2/z)-\arctan(y_3/z))
\end{equation*}
Two cases must be considered to determine the types of collisions possible: $t_r>t_3$ and $t_r<t_3$. These two
scenarios may lead to left edge ($\overline{23}$) and right edge ($\overline{14}$) collisions, respectively. Leading
edge ($\overline{12}$) collisions are possible and trailing edge ($\overline{34}$) collisions are impossible in both
cases.

\subsubsection{Right Edge Collisions}
For the set of parameters used in Fig \ref{fig.Plus15}, if the rotor is to the left of corner 2 as the bird enters the
rotor plane, it is impossible for the left edge of the bird ($\overline{23}$) to overtake the rotor, resulting in a
collision. Since $t_r<t_3$, the rotor is traveling to the left at a faster velocity than the right edge of the bird is
sweeping through the rotor plane. The green circle indicates the most advanced position of the rotor as the bird enters
the rotor plane that will result in a collision. This collision will occur with corner 2 of the bird at the instant the
bird enters the rotor plane. The point is given by
\begin{eqnarray}
    \nonumber y_- &=& y_2\\
    &=& y_{nose}-\frac{w \cdot cos(\phi)}{2}-t_{nose}V_{by}
\end{eqnarray}
The rotational angle associated with the green circle is $\theta_+$ (see Fig. \ref{fig.DeadOn}) and is given by
\begin{equation}
    \theta_+ = \arctan\left(\frac{z}{y_{nose}-\frac{w \cdot cos(\phi)}{2}-t_{nose}V_{by}}\right)
\end{equation}

Since the right edge of the bird ($\overline{14}$) also sweeps through the rotor plane slower than the rotor to the
left, it is possible for the rotor to clip corner 4 of the bird. Therefore, the last possible collision with the bird
occurs at the position of the black circle as corner 4 passes through the rotor plane. This point is given by
\begin{eqnarray}
    \nonumber y_+ &=& y_4\\
    &=& y_1-l \cdot sin(\phi) + (t_4-t_1)V_{by}
\end{eqnarray}
The rotation angle associated with $y_-$ is $\theta_-$ (see bottom pane of Fig. \ref{fig.DeadOn}) and is given by
\begin{equation}
    \theta_- = \arctan\left(\frac{z}{y_1-l \cdot sin(\phi) + (t_4-t_1)V_{by}}\right)
\end{equation}
The collision probability can then be calculated using (\ref{DownwindProbability}).

The same analysis applies for $z<0$, $\phi<0$ in downwind flight and $z>0$, $\phi<0$ in upwind flight except the corner
definitions must be swapped to account for the negative angle of approach where corner 1 is the first to intercept the
rotor plane.

\subsubsection{Left Edge Collisions} For the set of parameters used in Fig \ref{fig.Plus85}, the increased angle of
approach increase the speed at which the left edge of the bird ($\overline{23}$) sweeps through the rotor plane, such
that the rotor takes more time to travel the horizontal distance swept by this leading edge ($t_r>t_3$). This makes it
possible for a collision to occur between the rotor and the left edge of the bird. The green circle indicates the
initial position of the rotor (at $t=0$) that will clip corner 3 of the bird as the bird passes through the rotor
plane. The point is given by
\begin{equation}
    y_+ = \tan(\arctan(y_3/z)+t_3\Omega)z
\end{equation}
The rotational angle associated with the green circle is $\theta_+$ (see Fig. \ref{fig.DeadOn}) and is given by
\begin{equation}
    \theta_+ = \arctan\left(\frac{z}{\tan(\arctan(y_3/z)+t_3\Omega)z}\right)
\end{equation}

Since the right edge of the bird ($\overline{14}$) also sweeps to the left through the rotor plane faster than the
rotor, it is impossible for the rotor to clip corner 4 of the bird. Therefore, the last possible collision with the
bird occurs at the position of the black circle as corner 1 passes through the rotor plane. This point is given by
\begin{eqnarray}
    \nonumber y_- &=& y_1\\
    &=& y_{nose}+\frac{w \cdot cos(\phi)}{2}+(t_1-t_{nose})V_{by}
\end{eqnarray}
The rotation angle associated with $y_-$ is $\theta_-$ (see bottom pane of Fig. \ref{fig.DeadOn}) and is given by
\begin{equation}
    \theta_- = \arctan \left(\frac{z}{y_{nose}+\frac{w \cdot cos(\phi)}{2}+(t_1-t_{nose})V_{by}}\right)
\end{equation}
The collision probability can then be calculated using (\ref{DownwindProbability}).

The same analysis applies for $z<0$, $\phi<0$ in downwind flight and $z>0$, $\phi<0$ in upwind flight except the corner
definitions must be swapped to account for the negative angle of approach where corner 1 is the first to intercept the
rotor plane.

\subsubsection{Oblique Approach Turbine Collision Probabilities}
Figs. \ref{fig.Minus7Turbine}-\ref{fig.Plus85Turbine} show the collision probability contours for the four approach
angles described above ($-7^\circ$, $-15^\circ$, $15^\circ$, and $85^\circ$), respectively. For all of these cases, the
bird's air velocity $V_b$ is 5 m/s, the wind velocity $V_w$ is 10 m/s, and the other model parameters are as specified
in Table \ref{table.example_parameters}. In Fig. \ref{fig.Minus7Turbine} ($-7^\circ$ approach), it is apparent that the
contours are not symmetric about the horizontal axis. Since the bird is moving to the left, it is traveling in the
direction of the rotors in the top half of the turbine and against the direction of the rotors in the bottom half. As a
result, the probability of a collision is higher in the bottom half. In Fig. \ref{fig.Minus15Turbine} ($-15^\circ$
approach), this effect is amplified. Also note that the collision probabilities are higher in both halves for this case
since the oblique angle of approach means that the birds spend more time in the plane of the rotors. In Fig.
\ref{fig.Plus15Turbine} ($15^\circ$ approach), the probabilities for a $15^\circ$ approach are identical to a
$-15^\circ$ approach flipped about the horizontal axis of the turbine. This is because the bird is now moving against
the direction of the rotor rotation in the top half and with the direction of the rotor rotation in the bottom half. In
Fig. \ref{fig.Plus85Turbine} ($85^\circ$ approach), we can see that the highly oblique angle of the flight path
($25.5^\circ$ after accounting for wind velocity) results in a high probability of collision ($>0.5$) across the entire
face of the turbine. Also note, however, that the profile of the turbine, as viewed by the bird, is reduced, resulting
in a more narrow flight path that may result in a collision.

\begin{figure}
   \centering
   \includegraphics[width=1.0\columnwidth]{Minus7Turbine}
   \caption{Probability contours for a downwind flight path where the approach angle is $-7^\circ$ (see Fig. \ref{fig.Minus7}), bird's air velocity $V_b$ is 5
m/s, the wind velocity $V_w$ is 10 m/s, and the other model parameters are as specified in Table
\ref{table.example_parameters}}
   \label{fig.Minus7Turbine}
   \end{figure}

\begin{figure}
   \centering
   \includegraphics[width=1.0\columnwidth]{Minus15Turbine}
   \caption{Probability contours for a downwind flight path where the approach angle is $-15^\circ$ (see Fig. \ref{fig.Minus15}), bird's air velocity $V_b$ is 5
m/s, the wind velocity $V_w$ is 10 m/s, and the other model parameters are as specified in Table
\ref{table.example_parameters}}
   \label{fig.Minus15Turbine}
   \end{figure}

\begin{figure}
   \centering
   \includegraphics[width=1.0\columnwidth]{Plus15Turbine}
   \caption{Probability contours for a downwind flight path where the approach angle is $15^\circ$ (see Fig. \ref{fig.Plus15}), bird's air velocity $V_b$ is 5
m/s, the wind velocity $V_w$ is 10 m/s, and the other model parameters are as specified in Table
\ref{table.example_parameters}}
   \label{fig.Plus15Turbine}
   \end{figure}

\begin{figure}
   \centering
   \includegraphics[width=1.0\columnwidth]{Plus85Turbine}
   \caption{Probability contours for a downwind flight path where the approach angle is $85^\circ$ (see Fig. \ref{fig.Plus85}), bird's air velocity $V_b$ is 5
m/s, the wind velocity $V_w$ is 10 m/s, and the other model parameters are as specified in Table
\ref{table.example_parameters}. Note how the area swept out by the turbine rotors is becoming distorted due to the
oblique angle of approach. This is how the bird would view it as a result of the combined bird and wind velocity.}
   \label{fig.Plus85Turbine}
   \end{figure}


%\begin{figure*}
%   \centering
%   \includegraphics[width=2.0\columnwidth]{WindFarmRisk}
%   \caption{}
%   \label{fig.WindFarmRisk}
%   \end{figure*}

\section{Discussion}
The modeling of avian collision risk is an important step in the responsible deployment of wind farms. Since it has
been shown that the mortality rate resulting from collisions between birds and wind turbines is highly dependent on the
geology, turbine construction, and local bird species and behaviors, a model which can take into account all of these
factors can provide useful information for the planning of these installations and can significantly reduce the impact
on the affected bird populations.

This paper has described a kinematic model for calculating collision probabilities between birds and wind turbine
rotors. Due to its exclusion of any behavioral interactions between the birds and the turbines, this model alone is of
limited usefulness. Rather, it is intended as a primary building block of a larger model which may account for
behavioral interactions, populations of birds, and arrays of turbines in a wind farm.

Any accurate inclusion of the complex behavior interactions in play is a non-trivial task. Avoidance behaviors vary by
species, activity (e.g., foraging or migration), light and weather conditions, and perhaps by individual. These
behaviors may be arbitrarily complex, ranging from long distance avoidance of the wind farm to short distance avoidance
of an approaching rotor blade. Due to this complexity, it is common to use a global estimated avoidance rate to
calculate the collision probabilities \cite{Cooper2004}\cite{Podolsky2005}. These avoidance rates are often estimated
to be from 95\%-99.9\%, meaning that, at most, 5 of every 100 birds that are predicted to be hit by the model will
actually be hit. A better estimate of these avoidance rates may be discernable by studying the species in question at a
similar wind farm, but this is a complicated process and it is unclear how well these results may transfer to a new
site. While the range of avoidance rates generates a corresponding range of mortality estimates, a worst case scenario
may be acceptable for an impact assessment.


\bibliography{./references}
\bibliographystyle{IEEETran}

\end{document}
